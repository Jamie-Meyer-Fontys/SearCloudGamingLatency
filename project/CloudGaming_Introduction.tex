\section{Introduction}

\subsection{What is a cloud gaming service?}
This is the introduction written in the introduction.tex file!

\subsection{How do we define latency in this context?}
Latency in this research is defined as the time that it takes for the client to pass data/input, or a data packet from their place to another. Latency can be affected or determined by many possible factors, such as: Internet connection speed, the distance between the client and host, internet traffic, servers that are overloaded, the connection between the client's console and router, router settings not optimized for gaming, low memory space or old system hardware.\\\\
Whether the client is using a wired or a wireless network, both types share latency, however, the difference is noticeable. While a normal wired network doesn't quite travel at the speed of light, the latency is almost imperceptible to humans, being a 12-34 ms delay (depending on whether it's fiber or cable internet). When a wireless network is being used by the client, the latency is depended on the type of GHz that is being used, which in this case is 2.4 and 5 GHz. Where 2.4 GHz can be used in a large covered area and having wall penetration, the low data rate is one of the key points where latency can occur. And while 5 GHz has a smaller covered area where you can use it with no wall penetration, it uses Orthogonal frequency division multiplexing (OFDM) which creates a high data rate that keeps latency lower than what a 2.4 GHz network type can create.\\\\
In general, latency is a bad thing, to some extend. For people who don't normally play video games or online games and use their network (mostly) for social media, Internet searches, or messaging services, it's usually not a big problem. But for the people who do use their network to play video games (and especially games that require an Internet connection), it is disruptive. In an online video game, it is convenient (or even mandatory) for the player to play without internet interference, to perform the right actions at the right time multiple times. But when the latency in an online game is high, the player is often stuck on the frame where he/she was a few seconds ago and has (almost) no flexibility in their game. This is only for games that require an internet connection, where games that doesn't need one can run smoothly without latency issues (depending on your hardware).\\\\
So what about cloud gaming, a game that doesn't require an internet connection to play just runs without problems right? Unfortunately the problems are exactly the same as with an online game that is not attached to cloud gaming. Even if you play an offline game via cloud gaming, it still means that you can experience input latency. For example, where the player presses a button that is seen as a jump function, it might take more than 1-2 seconds for the input to be received by the host and the player, which can make the gameplay very annoying and difficult.
\subsection{Popular cloud gaming services}
While cloud gaming would not be the first option for most video game players, mainly because most prefer a video game console or gaming computer to have it in their homes, some companies have managed to make cloud gaming useable for the customer. Cloud gaming services such as Shadow, Geforce Now, Google Stadia, Project xCloud and Playstation Now have been well received by customers in the year 2021.\\\\
\textbf{Shadow}\\
Shadow is a cloud gaming service that is available for Windows, Mac, smartphone, tablet and for the television. The service requires a strong internet connection with 15 MBps at the minimum for the best performance.\\\\
To use the cloud gaming service of Shadow, you need to pay for a monthly subscription provided by Shadow, which allows access to their advanced hardware to stream the video games. Shadow gives the customer a high-end gaming computer for Windows 10 that is stored in their data center, which can be accessed anywhere. The computer is primarily made for the purpose to play video games that requires expensive hardware, but that's not all the computer has to offer. It can also be used for homework, video/sound editing, 3D rendering, and using other streaming services (Netflix, Hulu, etc.), which serves as an alternative to your personal computer.\\\\
What customers find very pleasing with Shadow, is how Shadow configures your environment/PC, that it offers responsive connection and gameplay and that many games are perfectly supported for the service.\\Though it has pleasing features, some customers still have downsides with it. The service itself isn't cheap, has some limited availability and some people find the hardware that is being used for the service not worth it for the monthly fee.\\\\
\textbf{Geforce Now}\\\\
\textbf{Google Stadia}\\\\
\textbf{Project xCloud}\\\\
\textbf{Playstation Now}