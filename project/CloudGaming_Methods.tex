\section{Methods}
This is the methods section written in the methods.tex file!

\subsection{Survey on (cloud) gaming}
We used Google's form app to create our own survey with questions about cloud gaming and personal gaming information. The survey we designed was completely anonymous. The personal questions were about what kind of system(s) the interviewee mainly used for video games, what kind of network was used, and what kind of video games the interviewee mainly played (single player, online multiplayer, co-op). Whereas the questions about cloud gaming focused more on what the interviewee thinks about cloud gaming, whether they were familiar with it, status of their internet connection and if they are interested in it.
\\\\
The audience we chose to fill out this survey were the students of Software Engineering and Business Informatics Fontys Venlo. Furthermore, we sent the survey to some social media groups that we were connected to.
\\\\
\subsection{Look up research articles related to cloud gaming}
To know more about our research question, we took the opportunity to search for multiple research articles that were related to cloud gaming and latency. To search these articles, we used search engines like Google and DuckDuckGo to find the possible articles we could use information from. All the articles that we have found and used, we immediately added the articles as references to our research. We used Google Scholar to create the citings and bibliography/references with BibTex.
\\\\
\subsection{RStudio and R}
RStudio is an open source software application that allows the client to create diagrams and charts that visualizes data from several datasets that you can use to import into the program or project. The application uses the programming language R, which is a language that is primarily used for data science and statistics. We used this program for our survey to visualize the data that we acquired from our audience. We didn't want to use any diagrams that Google Forms would automatically generate for us, since we still wanted to modify the data to make visualisations easier, and to us, Google doesn't visualize your survey answers properly.