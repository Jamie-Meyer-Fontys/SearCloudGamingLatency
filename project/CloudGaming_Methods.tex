\section{Methods}
This is the methods section written in the methods.tex file!

\subsection{Survey on (cloud) gaming}
We thought that public opinions could be useful for this research, so that's why we decided to create a survey with questions about cloud gaming and personal gaming information. The survey we designed was completely anonymous. The survey was divided in two parts: The first part focused on questions that were related to gaming from the perspective of the interviewee, where the second part focused on questions about cloud gaming and their opinion about its features, subscriptions and requirements.
\\\\
The audience we chose to fill in the survey were the students of Software Engineering and Business Informatics from Fontys Venlo. Furthermore, we sent the survey to some social media groups that we were connected to. The entire survey was built in Google Forms, an application tool developed by Google to create an online survey with ease. To get closer to our research question, we had to think of questions related to our topic. 
Questions such as what type of network, video games and system(s) the interviewee is related to were useful to us. There were also questions about network latency and whether it happened often at home. Finally, we also added questions to the survey to see if the interviewee is familiar with cloud gaming, if they know of certain services and what opinions they have about cloud gaming. When the survey is fully answered Google auto-generates graphs based on the answers, we didn't use this because we wanted to export and import the data to RStudio to analyse and visualize the data ourselves.
\\\\
\subsection{Look up research articles related to cloud gaming}
To learn more about our research question, we took the opportunity to search for multiple research articles that were related to cloud gaming and latency. To search these articles, we used search engines such as Google and DuckDuckGo to find the possible articles we could use information from. Any articles we found useful, we immediately added as references to our research. We used Google Scholar to create the citations and BibTex in TeXstudio to create the bibliography/references. We wanted to find as many articles as possible that were formatted or saved as PDFs and often mentioned in our search queries that the file type must be a PDF. Why we chose PDF primarily as a search query had mainly to do with the fact that instead of pdf files, we were getting more web pages back as search results with no downloadable pdf format. Specifying pdf format in the search queries made our work much more acceptable.
\\\\
\subsection{RStudio and R}
RStudio is an open source software application that allows the client to create diagrams and charts that visualizes data from several datasets that you can use to import into the program or project. The application uses the programming language R, which is a language that is primarily used for data science and statistics. We used this program for our survey to visualize the data that we acquired from our audience. We didn't want to use any diagrams that Google Forms would automatically generate for us, since we still wanted to modify the data to make visualisations easier, and to us, Google doesn't visualize your survey answers properly.